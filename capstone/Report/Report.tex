
%%%%%%%%%%%%%%%%%%%%%%% file typeinst.tex %%%%%%%%%%%%%%%%%%%%%%%%%
%
% This is the LaTeX source for the instructions to authors using
% the LaTeX document class 'llncs.cls' for contributions to
% the Lecture Notes in Computer Sciences series.
% http://www.springer.com/lncs       Springer Heidelberg 2006/05/04
%
% It may be used as a template for your own input - copy it
% to a new file with a new name and use it as the basis
% for your article.
%
% NB: the document class 'llncs' has its own and detailed documentation, see
% ftp://ftp.springer.de/data/pubftp/pub/tex/latex/llncs/latex2e/llncsdoc.pdf
%
%%%%%%%%%%%%%%%%%%%%%%%%%%%%%%%%%%%%%%%%%%%%%%%%%%%%%%%%%%%%%%%%%%%


\documentclass[runningheads,a4paper]{llncs}

\usepackage{amssymb}
\setcounter{tocdepth}{3}
\usepackage{graphicx}

\usepackage{url}
\urldef{\mailsa}\path|rparundekar@gmail.com|
\newcommand{\myTitle}{Understanding `Things' using Semantic Graph Classification} 
\newcommand{\myName}{Rahul Parundekar} 

\newcommand{\keywords}[1]{\par\addvspace\baselineskip
\noindent\keywordname\enspace\ignorespaces#1}

\begin{document}

\mainmatter  % start of an individual contribution

% first the title is needed
\title{Capstone Project: \myTitle}

% a short form should be given in case it is too long for the running head
%\titlerunning{Understanding DBpedia Using Graph Classification}

% the name(s) of the author(s) follow(s) next
%
% NB: Chinese authors should write their first names(s) in front of
% their surnames. This ensures that the names appear correctly in
% the running heads and the author index.
%
\author{\myName}
%
\authorrunning{\myName}
% (feature abused for this document to repeat the title also on left hand pages)

% the affiliations are given next; don't give your e-mail address
% unless you accept that it will be published
\institute{
Machine Learning Nanodegree, Udacity\\
\mailsa\\
\url{https://github.com/rparundekar}}

%
% NB: a more complex sample for affiliations and the mapping to the
% corresponding authors can be found in the file "llncs.dem"
% (search for the string "\mainmatter" where a contribution starts).
% "llncs.dem" accompanies the document class "llncs.cls".
%

\toctitle{\myTitle}
\tocauthor{\myName}
\maketitle


\begin{abstract}
The abstract should summarize the contents of the paper and should
contain at least 70 and at most 150 words. It should be written using the
\emph{abstract} environment.
\keywords{Ontology, Semantic Web, Graph Kernels, Graph Classification, Deep Learning}
\end{abstract}

\section{Definition}
\subsection{Project Overview}
The world around us contains different types of things (e.g. people, places,
objects, ideas, etc.). Predominantly, these things are defined by their attributes
like shape, color, etc. These things are also defined by their relationships with 
other things. For example, Washington D.C. is a
place and U.S.A is a country. But they have a relationship of Washington D.C.
being the capital of U.S.A., which adds extra meaning to Washington D.C. This
same role is played by Paris for France.

The Semantic Web is defined as ``an extension of the current Web in which
information is given well-defined meaning, better enabling computers and people to
work in cooperation"\cite{berners2001semantic}. 
It is an extention of the Knowledge Representation and Reasoning topic within Artificial Intelligence 
and deals with representing these things, their types, their attributes and their relationships
using symbols and enabling the agent to reason about them. 
A common data structure used for the representation is graphs. In these semantic graphs the nodes,
properties, and edges of graphs are very well suited to describe the things, their attributes,
and their relationships of things in the domain. A few example domais where such graphs are used are:
\begin{itemize}
\item Linked Data - The web-scale semantic data graph that is part of the Semantic Web\cite{heath2011linked}.
\item Spoken systems - the output of Natural Language Processing is a parse
tree \cite{socher2011parsing}.
\item Social networks are graphs \cite{backstrom2011supervised}.
\item Scene recognition - High level semantic information in images are graphs of arrangements of
things can be a graph \cite{socher2011parsing}.
\item Virtual \& Augmented Reality environments can be represented as a semantic graphs \cite{lugrin2007making}.
\end{itemize}

\subsubsection{Dataset:}
We use DBpedia\footnote{http://wiki.dbpedia.org/} as an exemplary dataset 
as a starting point to study Semantic Graph Classification. DBpedia is a 
large-scale knowledge base extracted from Wikipedia\cite{lehmann2015dbpedia}. 
It contains structured information extracted 
out of Wikipedia (e.g. page links, categories, infoboxes, etc.)\cite{dbpedia-swj}.
The semantic data in DBpedia can be represented as a graph of nodes and edges.
In this case, the nodes are \textit{things} (i.e. entities) and the edges are links/relationships between the
\textit{things}. Each \textit{thing} has one or more \textbf{types} \& \textbf{categories} associated with it.
The user community creating DBpedia maintains an Ontology\footnote{An Ontology is defined as a formal specification of the types, properties, and relationships of the entities that exist for a particular domain. In other words, it is the schema definition of the semantic data.} that specifies these types
of each of the \textit{things}. 

We use the subset of DBpedia\footnote{http://wiki.dbpedia.org/downloads-2016-04}, which was generated from the March/April 2016 dump of Wikipedia. In this dataset, the \textit{things}, their attributes and 
their relationships are extracted from the info-boxes (\texttt{infobox\_properties\_en.ttl}) using and the DBpedia Ontology (\texttt{dbpedia\_2016-04.owl}). We also use the types associated with the \textit{things}(\texttt{instance\_types\_en.ttl}) and the categories (\texttt{article\_categories\_en.ttl}) that they belong to.

\subsubsection{Project Motivation:}
The project is motivated by 
our assumption that if an Agent is able to classify things by understanding
its attributes and relationships into some types, we could in the future 
generalize it to an Agent that can act on the meaning of the things. 
Some examples applications for domains above are:
\begin{itemize}
\item Linked Data - Agents can understand and automatically assist users at web scale \cite{berners2001semantic}.
\item Spoken systems - Understanding user intent by Virtual Assistants like Siri, Alexa, etc. for home automation \cite{tang2017emergence}. 
\item Predicting and Recommending links in Social Networks \cite{backstrom2011supervised}.
\item Scene recognition - Urban scene understanding and its possible outdoor applications like understanding traffic, etc. \cite{cordts2015cityscapes}.
\item Virtural Assistants like Mara \cite{schmeil2007mara} in Virtual \& Augmented Reality environments.
\end{itemize}

Our goal is to try and estimate the types and categories of the \textit{things} from their attributes and relationships. 
For example, if you look at examples of categories in DBpedia, Achilies has been
put into the categories - demigods, people of trojan war, characters in Illead, etc.
What makes him part of those categories? Can we learn the definitions of these
based on the attributes and relationships of Achilies? 

\subsubsection{Other Approaches:}

\subsection{Problem Statement}
As mentioned earlier, the goal of this project is to apply Machine Learning 
to classify the \textit{things} in DBpedia and identify their types and categories
based on the semantic graph of 
their attributes andtheir relationships. 
To model this as a classification problem, we look at our algorithms
available in Machine Learning. The classic algorithms in Machine Learning
deal with feature vectors (e.g. the features used for classification, etc.) and are aimed
at essentially discriminating between different inputs to those features to identify the target type/s. 

%To make decisions based on the state of the world an A.I. Agent can
%read from the world using sensors etc., can easily perform a classification task
%once it learns the relation between the data to its output decision. For the most
%part, the feature vectors used in such cases as input encode the attributes of
%the things, BUT not necessarily the relationships between things. And while
%the designer of the inputs and outputs of the algorithms may manually craft
%features to represent some of these relationships, the Agent has no automatic
%way of comprehending and using these relationships.
\subsubsection{Graph Classification:}

\subsubsection{Proposed Approach:}

\subsubsection{Expected Result:}

\section{Preprocessing}

\bibliography{reportBib}
\bibliographystyle{splncs}

\end{document}
s