
%%%%%%%%%%%%%%%%%%%%%%% file typeinst.tex %%%%%%%%%%%%%%%%%%%%%%%%%
%
% This is the LaTeX source for the instructions to authors using
% the LaTeX document class 'llncs.cls' for contributions to
% the Lecture Notes in Computer Sciences series.
% http://www.springer.com/lncs       Springer Heidelberg 2006/05/04
%
% It may be used as a template for your own input - copy it
% to a new file with a new name and use it as the basis
% for your article.
%
% NB: the document class 'llncs' has its own and detailed documentation, see
% ftp://ftp.springer.de/data/pubftp/pub/tex/latex/llncs/latex2e/llncsdoc.pdf
%
%%%%%%%%%%%%%%%%%%%%%%%%%%%%%%%%%%%%%%%%%%%%%%%%%%%%%%%%%%%%%%%%%%%


\documentclass[runningheads,a4paper]{llncs}

\usepackage{amssymb}
\setcounter{tocdepth}{3}
\usepackage{graphicx}

\usepackage{url}
\urldef{\mailsa}\path|rparundekar@gmail.com|
\newcommand{\myTitle}{Understanding `Things' using Semantic Graph Classification} 
\newcommand{\myName}{Rahul Parundekar} 

\newcommand{\keywords}[1]{\par\addvspace\baselineskip
\noindent\keywordname\enspace\ignorespaces#1}

\begin{document}

\mainmatter  % start of an individual contribution

% first the title is needed
\title{Capstone Project: \myTitle}

% a short form should be given in case it is too long for the running head
%\titlerunning{Understanding DBpedia Using Graph Classification}

% the name(s) of the author(s) follow(s) next
%
% NB: Chinese authors should write their first names(s) in front of
% their surnames. This ensures that the names appear correctly in
% the running heads and the author index.
%
\author{\myName}
%
\authorrunning{\myName}
% (feature abused for this document to repeat the title also on left hand pages)

% the affiliations are given next; don't give your e-mail address
% unless you accept that it will be published
\institute{
Machine Learning Nanodegree, Udacity\\
\mailsa\\
\url{https://github.com/rparundekar}}

%
% NB: a more complex sample for affiliations and the mapping to the
% corresponding authors can be found in the file "llncs.dem"
% (search for the string "\mainmatter" where a contribution starts).
% "llncs.dem" accompanies the document class "llncs.cls".
%

\toctitle{\myTitle}
\tocauthor{\myName}
\maketitle


\begin{abstract}
The abstract should summarize the contents of the paper and should
contain at least 70 and at most 150 words. It should be written using the
\emph{abstract} environment.
\keywords{Ontology, Semantic Web, Graph Kernels, Graph Classification, Deep Learning}
\end{abstract}


\section{Introduction}
The world around us contains different types of things (e.g. people, places,
objects, ideas, etc.). Predominantly, these things are defined by their attributes
like shape, color, etc. These things are also defined by the ``roles” that they
play in their relationships with other things. For example, Washington D.C. is a
place and U.S.A is a country. But they have a relationship of Washington D.C.
being the capital of USA, which adds extra meaning to Washington D.C. This
same role is played by Paris for France.

The field of Knowledge Representation and Reasoning within Artificial Intelligence
deals with representing these things, types, attributes and relationships
using symbols and enabling the agent to reason about them. As it happens, a
convergence has come about in this field of knowledge representation from the
Databases domain - Graph databases can use graphs with nodes and edges to
represent data much similar to the knowledge graphs.

Many domains use semantic grahps to represent their information because nodes,
properties and edges of graphs are very well suited to describe the attributes
and relationships of things in the domain. For example:
\begin{itemize}
\item Spoken systems - the output of Natural Language Processing is a parse
tree.
\item Social networks are graphs.
\item High level semantic information in images are graphs of arrangements of
things.
\item The arrangement of objects on the road for autonomous driving is a graph.
\item A user’s browsing pattern of products, usage graphs, etc. is a graph (e.g. browing products, plan of actions, etc.).
\end{itemize}

In the classic sense, Machine Learning focuses on specific kinds of understanding
- classification, clustering, regression, etc. The algorightims in these deal with
feature vectors (e.g. the features used for classification, etc.) and are aimed
at essentially discriminating between different types of input to produce some
output. To make decisions based on the state of the world an A.I. Agent can
read from the world using sensors etc., can easily perform a classification task
once it learns the relation between the data to its output decision. For the most
part, the feature vectors used in such cases as input encode the attributes of
the things, BUT not necessarily the relationships between things. And while
the designer of the inputs and outputs of the algorithms may manually craft
features to represent some of these relationships, the Agent has no automatic
way of comprehending and using these relationships.

Can we use machine learning to make Agents better understand Things, including
their attributes AND their relationships? If we are able to inspect the attributes
and relationships of the things together and infer their roles, find its types, etc.
our agent can act on those. If an Agent is able to classify things by understanding
its semantic relationships, we could in the future generalize it to an Agent that
can act on the meaning of the things.

\subsection{Dataset}
An Ontology is a formalism used to represent semantic data about \textit{things}. It specifies the facts
about the \textit{things'} attributes and their relationships that the Agent believes to
be true. DBpedia\footnote{http://wiki.dbpedia.org/} is one such Ontology whose 
goal is to create a knowledge repository of general knowledge about the world 
such that Agents can have a grounding of the popular concepts and entities and 
their relationships. DBpedia datasets contains structured information extracted 
out of Wikipedia (e.g. page links, categories, infoboxes, etc.)\cite{dbpedia-swj}.

The semantic data in DBpedia can be represented as a graph of nodes and edges.
In this case, the nodes are \textit{things} (i.e. entities) and the edges are links/relationships between the
\textit{things}. Each \textit{thing} has both a \textbf{type} and a \textbf{category} associated with it.
Our goal is to try and estimate these values from the data about the \textit{thing}. 
We model this as a classification problem. An agent looking at graphical data 
can then use this classifier to identify the \textbf{type} or \textbf{category} of any \textit{thing} 
that it sees for the first time and then perform possible action for that \textit{thing}. 

We use DBpedia as an exemplary dataset as a starting point to study Semantic Graph Classification. 
This method could then be used in a variety of domains like the ones mentioned earlier. For example:
\begin{itemize}
\item Spoken systems - Have a dialog with the user based on the Natural Language Processing parse
tree.
\item Social networks - recommend interest groups to similar users.
\item Semantic image understanding - situation awareness.
\item Autonomous driving - perform the next driving action.
\item Product browsing pattern - recommend ads.
\end{itemize}

\subsection{Problem Statement}
Given the semantic data about things (i.e. their attributes and relationships),
can we identify their types (e.g. hierarchy of classes) or categories (e.g. roles it
plays) interest?

For example, if you look at examples of categories in DBpedia, Achilies has been
put into the categories - demigods, people of trojan war, characters in Illead, etc.
What makes him part of those categories? Can we learn the definitions of these
based on the attributes and relationships of Achilies?


\section{Preprocessing}

\bibliography{reportBib}
\bibliographystyle{splncs}

\end{document}
